\documentclass[12pt,letterpaper,titlepage]{article}
\usepackage{booktabs}
\usepackage[utf8]{inputenc}
\usepackage{amsmath}
\usepackage{amsfonts}
\usepackage{amssymb}
\usepackage{makeidx}
\usepackage{graphicx}
\usepackage{lmodern}
\usepackage{kpfonts}
\usepackage[left=1in,right=1in,top=1in,bottom=1in]{geometry}
\title{CS 540: Introduction to Artificial Intelligence\\Homework Assignment 4}
\author{Matthew Klebenow\\CSL: klebenow\\Section 1\\0 Late Days Used}
\date{November 30, 2012}
\begin{document}
\maketitle
\tableofcontents
\listoftables
\listoffigures
\pagebreak
\section[Question 1]{Search Algorithms}
Table~\ref{tab:q1} is a $5\times5$ maze filled with mouse traps $\left(\tau\right)$. There is a poor hungry mouse $\left(\mu\right)$ at one corner of the maze which smells the presence of cheese $\left(\xi\right)$ somewhere in the maze but does not know the exact location. Your task is to help the mouse find its food (fixed in square S) without getting trapped anywhere. Assume that the mouse can sense the presence of a mouse trap in neighboring cells. The mouse can only move in vertical and horizontal directions. It cannot move diagonally. Assume the successor function will cause \textit{legal moves to be examined in a clockwise order}: up, right, down, left. Note that not all of these moves may be possible from a given square.
\label{q1}
% Table generated by Excel2LaTeX from sheet 'q1'
\begin{table}[htbp]
  \centering
  \caption{Maze}
    \begin{tabular}{rrrrr}
    \toprule
    A     & B     & C     & D     & E \\
    $\mu$     &       & $\tau$     &       &  \\ \midrule
    F     & G     & H     & I     & J \\
          & $\tau$     &       &       &  \\ \midrule
    K     & L     & M     & N     & O \\
          &       &       &       & $\tau$ \\ \midrule
    P     & Q     & R     & S     & T \\
    $\tau$     &       & $\tau$     & $\xi$     &  \\ \midrule
    U     & V     & W     & X     & Y \\
          &       &       &       &  \\
    \bottomrule
    \end{tabular}%
  \label{tab:q1}%
\end{table}%
\subsection[Part 1]{Depth-First Search}
\label{1part1}
Using Depth-First Search, list the squares in the order they are expanded (including the goal node if it is found). Square A is expanded first (hint: State B will be examined next). Assume cycle checking is done so that a node is not generated in the search tree if the grid position already occurs on the path from this node back to the root node (i.e., Path Checking DFS). Write down the list of states you expanded in the order they are expanded. Write down the solution path found (if any), or explain why no solution is found.
\\\\
%% First attempt; didn't forget other routes
%First we have A. This expands to AB and AF. We expand AB and find no legal moves. We expand AF to find AFK. We expand AFK to find AFKL. We expand AFKL to find AFKLM and AFKLQ. We expand AFKLM to find AFKLMH and AFKLMN. We expand AFKLMH to find AFKLMHI. We expand AFKLMHI to find AFKLMHID and AFKLMHIJ. We cannot expand AFKLMHID. We expand AFKLMHIJ to AFKLMHIJE. We cannot expand AFKLMHIJE. We backtrack to AFKLMN. We expand AFKLMN to find AFKLMNS. We find $\xi$ at S. Therefore our solution path is as follows: \framebox{AFKLMNS}
%\subsubsection[Answer]{List of expanded states}
%\begin{enumerate}
%\item A
%\item AB
%\item AF
%\item AFK
%\item AFKL
%\item AFKLM
%\item AFKLMH
%\item AFKLMHI
%\item AFKLMHID
%\item AFKLMHIDE
%\item AFKLMHIJ
%\item AFKLMN
%\item \framebox{AFKLMNS}
%\end{enumerate}
\subsubsection[Answer]{List of expanded states}
A complete list of expanded states for the Path Checking Depth First Search is depicted in Table~\ref{tab:1_1ans}.
% Table generated by Excel2LaTeX from sheet '1-1'
\begin{table}[htbp]
  \centering
  \caption{Expanded states for Section~\ref{1part1}}
    \begin{tabular}{cc}
    \toprule
    Iteration & State \\
    \midrule
    1     & A \\ \midrule
    2     & AB \\ \midrule
    3     & AF \\ \midrule
    4     & AFK \\ \midrule
    5     & AFKL \\ \midrule
    6     & AFKLM \\ \midrule
    7     & AFKLMH \\ \midrule
    8     & AFKLMHI \\ \midrule
    9     & AFKLMHID \\ \midrule
    10    & AFKLMHIDE \\ \midrule
    11    & AFKLMHIDEJ \\ \midrule
    12    & AFKLMHIJ \\ \midrule
    13    & AFKLMHIJE \\ \midrule
    14    & AFKLMHIJED \\ \midrule
    15    & AFKLMN \\ \midrule
    16    & AFKLMNI \\ \midrule
    17    & AFKLMNID \\ \midrule
    18    & AFKLMNIDE \\ \midrule
    19    & AFKLMNIDEJ \\ \midrule
    20    & AFKLMNIJ \\ \midrule
    21    & AFKLMNIJE \\ \midrule
    22    & AFKLMNIJED \\ \midrule
    23    & AFKLMNIH \\ \midrule
    24    & AFKLMNS \\ \midrule
    \end{tabular}%
  \label{tab:1_1ans}%
\end{table}%
\subsection[Part 2]{Iterative Deepening Search}
\label{1part2}
Using Iterative Deepening Search, for each depth list the squares in the order they are expanded until a solution is reached. Use the same cycle checking as in $\S$\ref{1part1}.
\subsubsection[Answer]{List of expanded states}
\begin{enumerate}
\item A
\item ABF
\item ABFK
\item ABFKL
\item ABFKLMQ
\item ABFKLMQHNV
\item \framebox{ABFKLMQHNVISUW}
\end{enumerate}
In the last step, $I$ will be expanded to include $D$ and $J$. Next $S$ will be expanded and we will have reached the goal state.
\subsection[Part 3]{Manhattan Distance heuristic function}
\label{1part3}
Let each move (up/down/left/right) of the mouse have cost $1$. Consider the heuristic function $h(n)=\left|x_n-x_g\right|+\left|y_n-y_g\right|$, where the grid square associated with node $n$ is at coordinates $\left(x_n,y_n\right)$ on the board, and the goal node is at coordinates $\left(x_g,y_g\right)$. That is, $h(n)$ is the Manhattan distance between $n$ and the goal. Is $h(n)$ admissible? Why?
\subsubsection{Answer}
We define an admissible heuristic function to be one that never over-estimates the true cost from any given state to the goal state. The best possible situation we can place ourselves in is one where our current state has a clear shot to the goal; in this case, the heuristic would be equivalent to the true cost. In a worse situation where the mouse must avoid a mousetrap that is directly on its path to the goal, the heuristic under-estimates the true cost (the heuristic cost would be equivalent to charging straight through the mousetrap). As such, \framebox{the heuristic function $h\left(n\right)$ is admissible.}
\subsection[Part 4]{A$^*$ Search}
\label{1part4}
Regardless of your answer to \ref{1part3}, perform A$^*$ Search using the heuristic function $h(n)$ with a slight modification that $h(n)=\infty$ if node $n$ has a mouse trap. In the case of ties, expand states in alphabetical order. List each square in the order they are added to the OPEN list, and mark it with $f(n)=g(n)+h(n)$ (show $f$, $g$, and $h$ separately). Also, list the squares in the order they are expanded (including the goal node) and list the solution path found (if any), or explain why no solution is found.
\subsubsection[Answer]{Solution}
The process followed to reach the solution is shown explicitly in Table~\ref{tab:1part4}. The final solution path found is \framebox{AFKLMNS.}
% Table generated by Excel2LaTeX from sheet 'Sheet1'
\begin{table}[htbp]
  \centering
  \caption{Solution Steps for \ref{1part4}}
    \begin{tabular}{cccccc}
    \toprule
    Step  & OPEN  & $f=g+h$     & $g$     & $h$     & Expanded \\
    \midrule
    1     & A     & 6     & 0     & 6     &  \\
    2     &       &       &       &       & A \\
    3     & B     & 6     & 1     & 5     &  \\
    4     & F     & 6     & 1     & 5     &  \\
    5     &       &       &       &       & B \\
    6     &       &       &       &       & F \\
    7     & K     & 6     & 2     & 4     &  \\
    8     &       &       &       &       & K \\
    9     & L     & 6     & 3     & 3     &  \\
    10    &       &       &       &       & L \\
    11    & M     & 6     & 4     & 2     &  \\
    12    & Q     & 6     & 4     & 2     &  \\
    13    &       &       &       &       & M \\
    14    & H     & 8     & 5     & 3     &  \\
    15    & N     & 6     & 5     & 1     &  \\
    16    &       &       &       &       & Q \\
    17    & V     & 8     & 5     & 3     &  \\
    18    &       &       &       &       & N \\
    19    & I     & 8     & 6     & 2     &  \\
    20    & S     & 6     & 6     & 0     &  \\
    21    &       &       &       &       & S \\
    \bottomrule
    \end{tabular}%
  \label{tab:1part4}%
\end{table}%
\subsection[Part 5]{A$^*$ Search, heuristic function $h_2\left(n\right)$}
\label{1part5}
Repeat \ref{1part4} using a new heuristic function $h_2(n)$, which is defined using the old heuristic $h(n)$ as shown in Table~\ref{tab:h_2(n)}.
% Table generated by Excel2LaTeX from sheet '1-5'
\begin{table}[htbp]
  \centering
  \caption[$h_2(n)$]{New heuristic function $h_2(n)$}
    \begin{tabular}{rrrrrrrrrrr}
    \toprule
    h(n)  & 0     & 1     & 2     & 3     & 4     & 5     & 6     & 7     & 8     & infty \\
    \midrule
    h2(n) & 0     & 1     & 3     & 1     & 3     & 2     & 3     & 2     & 4     & infty \\
    \bottomrule
    \end{tabular}%
  \label{tab:h_2(n)}%
\end{table}%
A new mapping of the maze is shown in Table~\ref{tab:1part5maze} with each cell labeled with its name, attribute (mousetrap, cheese, mouse), and heuristic function values $h(n);h_2(n)$.
% Table generated by Excel2LaTeX from sheet '1-5'
\begin{table}[htbp]
  \centering
  \caption[Maze with heuristic values]{The maze from $\S$~\ref{q1} with values for both heuristic functions.}
    \begin{tabular}{ccccc}
    \toprule
    A     & B     & C     & D     & E \\
    $\mu$    &       & $\tau$   &       &  \\
    6;3   & 5;2   & $\infty;\infty$ & 3;1   & 4;3 \\\midrule
    F     & G     & H     & I     & J \\
          & $\tau$   &       &       &  \\
    5;2   & $\infty;\infty$ & 3;1   & 2;3   & 3;1 \\\midrule
    K     & L     & M     & N     & O \\
          &       &       &       & $\tau$ \\
    4;3   & 3;1   & 2;3   & 1;1   & $\infty;\infty$ \\\midrule
    P     & Q     & R     & S     & T \\
    $\tau$   &    & $\tau$   & $\xi$    &  \\
    $\infty;\infty$ & 2;3   & $\infty;\infty$ &  0;0  & 1;1 \\\midrule
    U     & V     & W     & X     & Y \\
          &       &       &       &  \\
    4;3   & 3;1   & 2;3   & 1;1   & 2;3 \\
    \bottomrule
    \end{tabular}%
  \label{tab:1part5maze}%
\end{table}%
\subsubsection[Answer]{Solution}
The process followed to reach the solution is shown explicitly in Table~\ref{tab:1part5ans}. The final solution path found is \framebox{AFKLMNS.}
% Table generated by Excel2LaTeX from sheet '1-4'
\begin{table}[htbp]
  \centering
  \caption{Solution steps for \ref{1part5}}
    \begin{tabular}{cccccc}
    \toprule
    Step  & OPEN  & $f=g+h$     & $g$     & $h$     & Expanded \\
    \midrule
    1     & A     & 3     & 0     & 3     &  \\
    2     &       &       &       &       & A \\
    3     & B     & 3     & 1     & 2     &  \\
    4     & F     & 3     & 1     & 2     &  \\
    5     &       &       &       &       & B \\
    6     &       &       &       &       & F \\
    7     & K     & 5     & 2     & 3     &  \\
    8     &       &       &       &       & K \\
    9     & L     & 4     & 3     & 1     &  \\
    10    &       &       &       &       & L \\
    11    & M     & 7     & 4     & 3     &  \\
    12    & Q     & 7     & 4     & 3     &  \\
    13    &       &       &       &       & M \\
    14    & H     & 6     & 5     & 1     &  \\
    15    & N     & 6     & 5     & 1     &  \\
    16    &       &       &       &       & H \\
    17    & I     & 9     & 6     & 3     &  \\
    18    &       &       &       &       & N \\
    19    & S     & 6     & 6     & 0     &  \\
    20    &       &       &       &       & S \\
    \bottomrule
    \end{tabular}%
  \label{tab:1part5ans}%
\end{table}%
\pagebreak
\section[Question 2]{Hill Climbing}
\label{q2}
We would like to solve the n-Queens problem using a greedy hill-climbing algorithm. The n-Queens problem required that we place all n queens on an $n \times n$ board so that none of them can attack any other by the rules of chess. This means there cannot be 2 queens in the same row, column, or diagonal. Here we define each state to correspond to a complete assignment of a row number from 1 to n to each of the n column variables $C_1$ through $C_n$. The successor operator Succ(s) generates all neighboring states of s, which we will define as all total assignments which differ by exactly one variable's row value. So, for example, given $n=2$ the state with assignments $\left\lbrace C_1=1,C_2=2\right\rbrace$ has two neighboring states $\left\lbrace C_1=2,C_2=2\right\rbrace$ and $\left\lbrace C_1=1,C_1=1\right\rbrace$ corresponding to moving either the queen in the first row or the queen in the second. When breaking ties, order the variables from left to right, and the values for the variables from bottom to top of the board and pick the neighboring state that comes first in the ordered list of states.
\subsection[Part 1]{Successor function and search space}
If you have $n$ rows and $n$ column variables, how many neighboring states does the $\texttt{Succ}\left(s\right)$ function produce? What is the total size of the search space?
\subsubsection{Answer}
For any given succession state, we may change any $C_i$ to a value from 1 to n, excluding the current value of $C_i$. We have $n$ columns to work with, and $n-1$ possible positions to change each column to. Therefore, the successor function will produce $n\times\left(n-1\right)$ neighboring states. The total size of the search space must then be the aforementioned branching factor times the depth of the problem space. Since the depth is not known a priori, the total search space size is unknown.
\label{2part1}
\subsection[Part 2]{State evaluation function}
Define an evaluation function for the states such that the goal state has the highest value when the evaluation function is applied to it.
\subsubsection{Answer}
\label{evalq}
Let us define an evaluation function based on the number of queens attacking each other, $x$. If we evaluate every state with a value of $-x$, the goal state will have a maximum value of $0$.
\label{2part2}
\subsection[Part 3]{6-Queens Plateau}
Consider the 6-Queens problem and come up with a non-goal state that is on a plateau  in our hill-climbing space using the evaluation function you defined above.
\subsubsection{Answer}
Similar to the 8-Queens example shown in class that is also on a plateau, the state shown in Figure~\ref{fig:6q} is a non-goal plateau state in our hill-climbing space using the evaluation function defined in $\S$~\ref{evalq}.
\begin{figure}[h]
\centering
\includegraphics[scale=1]{6q.jpg}
\caption{Plateau state}
\label{fig:6q}
\end{figure}
\label{2part3}
\section[Question 3]{Iterative Deepening}
\label{q3}
Consider a search space that consists of a tree with branching factor $b>1$, so each node has exactly $b$ unique children. The root node is at depth $0$. The first node checked on level $d>0$ is a solution to the search. How many nodes \textit{total} will an iterative deepening search check before it halts?
\subsection{Answer}
For the first iteration, we must check the root node. This action requires 1 check. The next iteration, we must check the root node and it's children. This action requires $1+b$ checks. The next iteration, we must check: the root ($1$), its children ($b$), and all of its children's children ($b^2$). The total number of checks for this iteration is therefore $1+b+b^2$. This pattern will continue until the iteration reaches the depth $d>0$ and checks $1+b+b^2+b^3+\cdots+b^{d-2}+b^{d-1}+1$ nodes. Note that the $1$ that terminates the sequence is because the first node checked on that iteration (the iteration at level $d$) is the solution to the search. So, the \textit{total} number of nodes checked by the iterative deepening search is the sum of checks performed across all iterations. A full mathematical interpretation of this result is given in Equation~\ref{eq:q3}.
\begin{multline}
\label{eq:q3}
\text{Total checks} = 1 + \left(1+b\right) + \left(1+b+b^2\right) + \left(1+b+b^2+b^3\right) + \left(1+b+b^2+b^3+b^4\right) + \cdots\\
\cdots + \left(1+b+b^2+b^3+b^4 +\cdots+b^{d-2}+b^{d-1}+1\right)
\end{multline}
\end{document}